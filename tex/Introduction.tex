\section{Introduction}

Le problème des 100 prisonniers se présente comme suit:
le directeur d'une prison décide d'offrir une chance à 100 prisonniers
condamnés à mort sous la forme d'une épreuve.
Chacun d'eux porte un uniforme numéroté de 1 à 100, dans 100 boîtes
numérotées de 1 à 100 et fermées, chacune contient un papier correspondant
au numéro d'un seul prisonnier.
Toutes ces boîtes sont placées dans une salle isolée. 

Les prisonniers doivent y entrer tour à tour et seuls.
Après le passage de chaque prisonnier, la salle est remise dans le même
état qu'elle était initialement.

Lors du passage dans cette salle, chacun d'eux à le droit d'ouvrir
et de regarder dans 50 boîtes au plus,

\begin{itemize}
    \item 
    si durant son tour un prisonnier parvient à trouver son propre numéro
    dans une boite, il peut alors quitter la salle.
    \item
    s'il ne pavient pas à trouver son propre numéro, alors tout les
    prisonniers sont exécutés.
\end{itemize}

Les prisonniers ne peuvent pas communiquer entre eux lors de l'épreuve,
mais il peuvent établir une stratégie avant qu'elle ne commence.\\\\
Ce problème est une version modifiée du problème formulé initialement par
Peter Bro Miltersen, et de la version de Philippe Flajolet et Robert Sedgewick.

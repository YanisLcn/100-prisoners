\section{Introduction}

Le problème des 100 prisonniers \cite{100PrisonersProblem2023} se présente
comme suit:
le directeur d'une prison décide d'offrir une chance à 100 prisonniers
condamnés à mort sous la forme d'une épreuve.
Chacun d'eux porte un uniforme numéroté de 1 à 100, et dans 100 boîtes
numérotées de 1 à 100 fermées, chacune contient un papier avec le numéro d'un
prisonnier correspondant écrit dessus.
Les papiers sont répartis aléatoirement dans les bôites et elles sont placées dans une
salle isolée.

Les prisonniers doivent y entrer tour à tour et seuls.
Après le passage de chaque prisonnier, la salle est remise dans le même
état qu'elle était initialement.

Lors du passage dans cette salle, chacun d'eux à le droit d'ouvrir
et de regarder dans 50 boîtes au plus,

\begin{itemize}
	\item
	      si durant son tour un prisonnier parvient à trouver son propre numéro
	      dans une boite, il peut alors quitter la salle.
	\item
	      s'il ne pavient pas à trouver son propre numéro, alors tout les
	      prisonniers sont exécutés.
\end{itemize}

Les prisonniers ne peuvent pas communiquer entre eux durant l'épreuve,
mais il peuvent établir une stratégie avant qu'elle ne commence.\\\\
Ce problème présenté \cite{veritasiumRiddleThatSeems2022} par
\href{https://fr.wikipedia.org/wiki/Derek_Muller}{Derek Muller}
(alias \href{https://www.youtube.com/@veritasium}{Veritasium})
est une version modifiée du problème
initial \cite{miltersenCellProbeComplexity2007} formulé par
\href{https://pure.au.dk/portal/en/persons/bromille%40cs.au.dk}{Peter Bro Miltersen}
, et de la version \cite{flajoletAnalyticCombinatorics2009} de
\href{https://fr.wikipedia.org/wiki/Philippe_Flajolet}{Philippe Flajolet}
et \href{https://en.wikipedia.org/wiki/Robert_Sedgewick_(computer_scientist)}{Robert Sedgewick}.

\section{Baby-Step Giant-Step}

La méthode utilisée pour la recherche de cycles dans l'énigme des 100 prisonniers peut
être appliquée dans le domaine de la cryptographie

\begin{tcolorbox}
    Un groupe $G$ est cyclique s'il existe $x \in G$ 
    tel que tout élément de $G$ est un multiple de $x$.
    On dit alors que $x$ est un générateur de $G$ 
\end{tcolorbox}

\begin{tcolorbox}[title=Algorithme \href{https://fr.wikipedia.org/wiki/Baby-step_giant-step}{Baby-Step Giant-Step}]
    \textbf{Entrées :} $g$ un groupe cyclique, $\alpha$ générateur de $g$, $\beta$ élément de $g$ \\
    \textbf{Sortie :} une valeur $x$ telle que $g^x \equiv \alpha \pmod{\beta}$

    \begin{enumerate}
        \item  Calculer $m = \sqrt{\beta} +1$.
        \item Baby-Step :
              \[
                \begin{aligned}
                    \text{pour } i = 0 \text{ à } m-1: \quad & \text{stocker à l'indice $i$ d'une liste $babyStep$ : } \alpha^i
                \end{aligned}
              \]
        \item Soit $\gamma = \beta$
        \item Giant-Step :
              \[
                \begin{aligned}
                    \text{pour } i = 0 \text{ à } m-1: \quad & \text{si } \gamma \text{ est égal à } babyStep[i] \\
                                                             & \text{alors retourner } i \cdot m + j \\
                                                             & \text{sinon } \gamma = \gamma \cdot \alpha^{-m}
                \end{aligned}
              \]
    \end{enumerate}

\end{tcolorbox}

Cet algorithme permet de trouver le \href{https://fr.wikipedia.org/wiki/Logarithme_discret}{logarithme discret} d'un groupe cyclique $G$,
c'est-à-dire le plus petit entier naturel $k$ tel que $x = b^k$ avec $b$ générateur de $G$
et $x$ élément quelconque de $G$.

Couplé à l'\href{https://fr.wikipedia.org/wiki/Exponentiation_rapide}{exponentiation rapide}, le logarithme discret sert à la cryptographie à clé publique,
la combinaison des deux objets produisant l'asymétrie recherchée dans le contexte de l'échange de clés.

Si on applique le même type de recherche des cycles (cf \href{https://github.com/YanisLcn/100-prisoners/blob/master/tools/permutation_cycles.py}{Annexe}) que pour le problème des 100 prisonniers,
puis que l'on applique l'algorithme Baby-Step Giant-Step sur les cycles optimaux trouvés, on obtient
alors leurs logarithmes discrets respectifs.
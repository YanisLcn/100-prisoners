\documentclass{article}
\usepackage[francais]{babel}

\title{100-prisonniers}
\date{27-11-2023}
\author{GUETTEVILLE Nathan | SOAN Tony Ly | LACENNE Yanis | G4S12}

\begin{document}
\pagenumbering{gobble}
\maketitle
\pagenumbering{arabic}

\tableofcontents

\newpage
\section{Introduction}
Présentation et description du problème des 100 prisionniers

TODO : Tony Ly

\section{Lien avec les mathématiques discrètes}
Faire le lien avec les graphes et les permutations.

\section{Probabilités}
Comment se fait-il que les prisonniers aient un pourcentage de chances de s'en sortie de 30\% ?

TODO : Yanis

\section{Baby-Step Giant-Step}
Qu'est-ce que l'algorithme `Baby-Step Giant-Step` ? À quoi sert-il et quel est le lien avec les prisonniers ?

TODO : Nathan

\section{Conclusion}

\section{Annexe}
Notre annexe numérique est susceptible de contenir
\begin{itemize}
	\item Programme : Trouver tout les cycles d'une permutation
	      (TODO : Tony Ly)
	\item Programme : Afficher le graphe d'une permutation
	      (TODO : Yanis)
	\item Programme : Tester la probabilité de survie des prisonniers / Comparaison entre la stratégie aléatoire et la stratégie gagnante.
	      (TODO : Nathan)
\end{itemize}

\end{document}

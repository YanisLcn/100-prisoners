\documentclass[a4paper]{article}
\usepackage[french]{babel}
\usepackage[T1]{fontenc}
\usepackage{authblk}
\usepackage{url}
\usepackage{hyperref}
\usepackage[nottoc,numbib]{tocbibind}

\title{100-prisonniers}
\date{27/11/2023}

\author{GUETTEVILLE Nathan}
\author{LACENNE Yanis}
\author{SOAN Tony Ly}

\affil{G4S12}

\renewcommand\Authands{ et }
\begin{document}

\maketitle

\pagenumbering{gobble}
\pagenumbering{arabic}

\tableofcontents

\newpage

\section{Introduction}

Le problème des 100 prisonniers \cite{100PrisonersProblem2023} se présente
comme suit:
"le directeur d'une prison décide d'offrir une chance d'être libéré à 100 prisonniers
condamnés à mort sous la forme d'une épreuve.
Chacun d'eux porte un uniforme numéroté de 1 à 100, et dans 100 boîtes distinctes
numérotées de 1 à 100 fermées, chacune contient un papier avec le numéro d'un
prisonnier correspondant écrit dessus.
Les papiers sont répartis aléatoirement dans les boîtes et elles sont placées dans une
salle isolée.

Les prisonniers doivent y entrer tour à tour et seuls.
Après le passage de chaque prisonnier, la salle est remise à l'état initial.

Lors du passage dans cette salle, chacun d'eux à le droit d'ouvrir
et de regarder dans 50 boîtes au plus,

\begin{itemize}
	\item
	      si durant son tour un prisonnier parvient à trouver son propre numéro
	      dans une boite, il peut alors quitter la salle.
	\item
	      s'il ne parvient pas à trouver son propre numéro, alors tous les
	      prisonniers sont exécutés.
\end{itemize}

Les prisonniers ne peuvent pas communiquer entre eux durant l'épreuve,
mais il peuvent établir une stratégie avant qu'elle ne commence."\\\\
Ce problème présenté \cite{veritasiumRiddleThatSeems2022} par
\href{https://www.veritasium.com/about}{Derek Muller}
(alias \href{https://www.youtube.com/@veritasium}{Veritasium})
est une version modifiée du problème
initial \cite{miltersenCellProbeComplexity2007} formulé par
\href{https://pure.au.dk/portal/en/persons/bromille%40cs.au.dk}{Peter Bro Miltersen}
, et de la version \cite{flajoletAnalyticCombinatorics2009} de
\href{https://fr.wikipedia.org/wiki/Philippe_Flajolet}{Philippe Flajolet}
et \href{https://sedgewick.io/}{Robert Sedgewick}.


\section{Lien avec les mathématiques discrètes}
Faire le lien avec les graphes et les permutations.

\section{Probabilités}
Comment se fait-il que les prisonniers aient un pourcentage de chances de s'en sortie de 30\% ?

TODO : Yanis

\section{Baby-Step Giant-Step}
Qu'est-ce que l'algorithme `Baby-Step Giant-Step` ? À quoi sert-il et quel est le lien avec les prisonniers ?

TODO : Nathan

\section{Conclusion}

\section{Annexe}
Notre annexe numérique est susceptible de contenir
\begin{itemize}
	\item Programme : Trouver tout les cycles d'une permutation
	      (TODO : Tony Ly)
	\item Programme : Afficher le graphe d'une permutation
	      (TODO : Yanis)
	\item Programme : Tester la probabilité de survie des prisonniers / Comparaison entre la stratégie aléatoire et la stratégie gagnante.
	      (TODO : Nathan)
\end{itemize}

\bibliographystyle{IEEEtran}
\bibliography{citation}

\end{document}
